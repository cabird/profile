\documentclass[9pt]{article}
% Seems like it does not support 9pt and less. Anyways I should stick to 10pt.
%\documentclass[a4paper, 9pt]{article}
\topmargin-2.0cm

\usepackage{fancyhdr}
\usepackage{pagecounting}
\usepackage[dvips]{color}
\usepackage{microtype}

\newenvironment{packed_enum}{%
\vspace{-4pt}
\begin{enumerate}
  \setlength{\itemsep}{1pt}
  \setlength{\parskip}{0pt}
  \setlength{\parsep}{0pt}
}{\end{enumerate}
\vspace{-4pt}%
}




\advance\oddsidemargin-1in
\textheight9.5in
\textwidth6.5in
\def\baselinestretch{1.0}
%\pagestyle{empty}

\newcommand{\hsp}{\hspace*{\parindent}}
\definecolor{gray}{rgb}{0.4,0.4,0.4}
%\definecolor{gray}{rgb}{1.0,1.0,1.0}

\newcommand\Section[1]{\vspace{6pt}\noindent\textbf{#1}\vspace{4pt}}
\newcommand\Subsection[1]{\subsubsection*{\small #1}}

\begin{document}
\thispagestyle{empty}
%\pagenumbering{gobble}
%\fancyhead[location]{text} 
% Leave Left and Right Header empty.
\lhead{}
\rhead{}
%\rhead{\thepage}
\renewcommand{\headrulewidth}{0pt} 
\renewcommand{\footrulewidth}{0pt} 

\begin{small}

%\vspace*{0.1cm}
\begin{center}
{\LARGE \bf Research Statement}\\
\vspace*{0.2cm}
{\large Christian Bird}\\
\vspace*{0cm}
\texttt{\normalsize cabird@ucdavis.edu}
\end{center}

As software
projects grow in size and complexity, so do the development teams.  As this
hateam structure and process  become critical determinants of programmer
productivity, as well as software quality.  Conway and others have argued that
good design and modularization leads to more effective division of labor and
knowledge in large, complex systems, and thus better quality and productivity
outcomes.  My work has been among the earliest to empirically study the
relationships between software characteristics, team dynamics, and software
engineering outcomes in commercial and open-source software projects.  

How are large software teams organized? What factors of team organization
influence productivity and quality?  I have examined the relationships between
software engineers and the code that they develop, contribute, discuss,
release, fix, and maintain, to ask and answer questions such as: When do
developers need to coordinate their changes?  What are the communication
patterns of the most productive developers?  How do contribution patterns in
large projects affect code quality? Is a team affected by geographic
distribution?  Long term, my research will provide empirically grounded
principles that govern the effective co-design of software and development
teams. 

I am also interested in the use of rigorous empirical methods to evaluate
questions about software engineering processes.  Over the last few years,
researchers in our field have begun applying more advanced, modern methods
developed in fields such as bio-informatics, econometrics, and organizational
behavior, e.g., statistical network analysis, hazard rate models, and principal
component regression.  More notably, these research trends are beginning to
influence the practice of software development in industry.  My research is at
the heart of these trends.


\Section{Research Plan}

Games and web applications are growing in importance, and are attracting
enormous investments in terms of time, money, and talent. However, there are
still as yet very few studies of development projects in these areas. Are these
domains like traditional software engineering? If not, how are they different?
The answers to these questions help us determine what principles are more
universal and which are specific to characteristics of software projects.  This
can, in turn, aid planning and governance of software projects by informing
stakeholders of how decisions may affect productivity and quality, and ultimately
time lines and cost.

Web applications are becoming more and more complex and prevalent.  The
architecture of these applications often spans a number of machines from the
client to a web server to a back-end data store.  Development of these
applications has many differences from traditional development.

\begin{packed_enum}

\item \emph{Live Hosting} - 
%
    In stark contrast to traditional development where a relatively static
    product is downloaded and installed on a user's machine, web applications
    are repeatedly ``deployed'' whenever they are accessed by a browser.  This
    gives development teams greater flexibility in terms of release cycles, 
    but also creates great risk, as a bug on one server can
    (and has) make an application unusable for the entire user-base.  I plan to
    investigate the opportunities and costs associated with this key
    difference.

\item \emph{Monitoring} - 
%
    Since web applications are hosted on servers, usage of the software can be
    monitored much more closely than traditional applications.  Crash reporting
    for desktop applications can provide the context of a program failure, but
    gives no indication of what parts of an application are used heavily and
    \textbf{not} failing, or even what proportion of users are experiencing
    problems.  I plan to explore uses for this data, such as resource allocation,
    feature planning.
    
\end{packed_enum}

The video game industry represents \$10's of billions annually. However, game
development differs from the development of typical products in some important
ways.  Are the principles of software engineering presented in prior research
affected by these differences?

\begin{packed_enum}

\item \emph{Content} - 
%
    Most games are split into an engine, which represents executable code, and
    data content. Compared to other software domains, a disproportionately
    large amount of money, time, talent, and space is devoted on content.
    I plan to explore the relationship between content and code
    relationships in an effort to devise new methods for bug prediction, team
    member awareness, and recommendation systems, which traditionally are
    dependent on source code attributes.

\item \emph{Team Makeup} - 
%
    A sizeable proportion of team members are not software developers,
    including artists, story line, and other content creators.  One might well
    expect that the communication and coordination needs of members in content
    creation and content testing roles differ from software developers.  I plan
    to observe, record, and mine interactions, socio-technical relationships,
    and information needs in order to characterize the differences in these
    roles and correlate them with outcomes, leading to an increased
    understanding of the team dynamics.

\item \emph{Closed Systems} -
%
    In the past most gaming software has run in a relatively closed system.
    Console games run on a very restricted set of hardware and the set of
    ``installed'' software is constrained.  However, this is changing as games
    are moving online, must support downloadable content, and are supporting
    more complex and varied input mechanisms.  Do these differences affect the
    testing needed to assure quality and stability?  

\end{packed_enum}

My research will include devising methods of gathering data
and applying both quantitative and qualitative empirical techniques 
to these domains.

\end{small}

\end{document}

