\documentclass[10pt]{article}
% Seems like it does not support 9pt and less. Anyways I should stick to 10pt.
%\documentclass[a4paper, 9pt]{article}
\topmargin-2.0cm

%\usepackage{fancyhdr}
%\usepackage{pagecounting}
%\usepackage[dvips]{color}

% Color Information from - http://www-h.eng.cam.ac.uk/help/tpl/textprocessing/latex_advanced/node13.html

% NEW COMMAND
% marginsize{left}{right}{top}{bottom}:
%\marginsize{3cm}{2cm}{1cm}{1cm}
%\marginsize{0.85in}{0.85in}{0.625in}{0.625in}

\advance\oddsidemargin-1in
%\advance\oddsidemargin-0.65in
%\advance\evensidemargin-1.5cm
\textheight9.4in
\textwidth6.5in
%\def\baselinestretch{0.99}
%  \pagestyle{empty}
\setlength{\parskip}{5pt} 

%\newcommand{\hsp}{\hspace*{\parindent}}
%\definecolor{gray}{rgb}{0.4,0.4,0.4}
%\definecolor{gray}{rgb}{1.0,1.0,1.0}

\begin{document}

\pagestyle{empty}

%\vspace*{0.1cm}
\begin{center}
{\LARGE \bf Teaching Statement}\\
\vspace*{0.2cm}
{\large Christian Bird}\\
\vspace*{0.1cm}
\texttt{\normalsize cabird@ucdavis.edu}\\
\end{center}
%\vspace*{0.2cm}

%\begin{document}
%\centerline {\Large \bf Research Statement for Sundar Iyer}
%\vspace{0.5cm}

% Write about research interests...
%\footnotemark
%\footnotetext{Check This}


Although I originally entered computer science because of my own interests,
I have had many opportunities to
teach others along the way.  As an undergraduate, I was a teaching assistant in
algorithms and data structures and in graduate school I taught the discussion
sections and held regular office hours for the undergraduate programming
languages course.  I have also given guest lectures for a number of graduate
courses.  Through these experiences, I've discovered that I genuinely enjoy teaching
things that I am passionate about.  After having taken classes from many
instructors, and observing others, as well as attending talks, invited
lectures, and keynotes, I have learned that an audience is engaged and retains
the most when the person presenting material is enthusiastic and passionate
about the subject.

My wife received an undergraduate degree in education and has been a teacher
for 6 years.  During my teaching responsibilities, I have had the benefit of
discussing strategies and methods with her.  For example, I've used in-class
discussions and exercises to address learning at the different levels of Bloom's
Taxonomy.  On one occasion, when a number of students were attending my office
hours due to confusion on the use of functional programming, I worked with one
student and then watched as she taught the concepts to other students who arrived
soon after.  Because she had to understand well enough to teach others,
\emph{and} because the other students asked insightful questions, all were able
to learn together in a memorable way.

I believe that understanding fundamentals is important, and that an application
of fundamentals aids students because it helps them see how they are used in
practice and most people learn better by doing than only reading or listening.
I have experienced this in my own career.  I found DFAs and NFAs confusing when
encountering them in my freshman year of college.  However, after studying
their uses in regular languages and writing a program to convert an NFA to a
DFA and execute it on an input string, I understood \emph{how} they worked,
\emph{when} they could and should be used, and \emph{why} one would want to
convert an NFA to a DFA.  While students learn in different ways and I plan to
accommodate all, I plan to begin by teaching fundamentals and following up with
practical exercises and discussion of current practice.

My research discipline is software engineering and I plan to teach in this
area.  As results from software engineering research transfer to practice in
industry, software engineers are needing new skills. Fresh graduates entering the
profession should be familiar not only with programming and software
development processes, but also empirical methods that can help improve
productivity and software quality.  Many recent research developments can
benefit software production environment, including: identifying the most
failure prone modules prior to deployment, determining what communication
channels should exist based on team and software structure, and identifying
cross-cutting concerns early in the life-cycle.  The Capability Maturity Model,
a metric by which software development organizations are measured, requires
that processes and outcomes be measured and analyzed.  These are skills that I
acquired after entering the work force and beginning graduate school, but which
\emph{should} be taught at the undergraduate level.  These skills will help
differentiate new graduates and will raise the quality of software engineering
in general.

While I look forward to teaching classes on software engineering at the
graduate and undergraduate level, my interests and knowledge in computer
science is more broad.  My minor emphasis in graduate school is in theory, and I
would enjoy and feel comfortable teaching classes on programming languages,
compilers, algorithms and data structures, and theory of computation as well as
introductory computer science courses.

I have been fortunate to work with an advisor who is not only a bright
researcher, but also a gifted mentor.  Prem has seen it as his responsibility
to help me become an independent and critical thinker.  My advisor has also
acted as a wonderful motivator.  On multiple occasions early in my career, I
would arrive at a weekly meeting disappointed with recent results.  In every case, I came
away re-energized, seeing the situation in a new light, and with new ideas to
move forward with.  I now have a peer relationship with my advisor and often
debate the merits of different ideas, results, and approaches as we discuss
papers we have read or research plans for the future.  I will use a similar
approach as I advise graduate students. Beyond assisting in teaching university
courses, I have had the opportunity to work with and mentor both undergraduates
and graduate students.  This process has been rewarding as I have been
able to help students as they tackle real world problems, face unforeseen
challenges, and finally write up their results for publication.  I will
continue to guide students in their own endeavors, whether they be careers in
research or practice.

\end{document}

