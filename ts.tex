\documentclass[10pt]{article}
% Seems like it does not support 9pt and less. Anyways I should stick to 10pt.
%\documentclass[a4paper, 9pt]{article}
\topmargin-2.0cm

\usepackage{fancyhdr}
\usepackage{pagecounting}
\usepackage[dvips]{color}

% Color Information from - http://www-h.eng.cam.ac.uk/help/tpl/textprocessing/latex_advanced/node13.html

% NEW COMMAND
% marginsize{left}{right}{top}{bottom}:
%\marginsize{3cm}{2cm}{1cm}{1cm}
%\marginsize{0.85in}{0.85in}{0.625in}{0.625in}

\advance\oddsidemargin-1in
%\advance\oddsidemargin-0.65in
%\advance\evensidemargin-1.5cm
\textheight9.2in
\textwidth6.5in
%\def\baselinestretch{1.05}
%\pagestyle{empty}

\newcommand{\hsp}{\hspace*{\parindent}}
\definecolor{gray}{rgb}{0.4,0.4,0.4}
%\definecolor{gray}{rgb}{1.0,1.0,1.0}

\newcommand\Section[1]{\subsubsection*{\large #1}}
\newcommand\Subsection[1]{\subsubsection*{\small #1}}

\begin{document}
%\thispagestyle{fancy}
%\pagenumbering{gobble}
%\fancyhead[location]{text} 
% Leave Left and Right Header empty.
\lhead{}
\rhead{}
%\rhead{\thepage}
\renewcommand{\headrulewidth}{0pt} 
\renewcommand{\footrulewidth}{0pt} 
%\fancyfoot[C]{\footnotesize \textcolor{gray}{http://wwwcsif.ucdavis.edu/$\sim$bird/index.html}}

%\pagestyle{myheadings}
%\markboth{Sundar Iyer}{Sundar Iyer}

%\pagestyle{fancy}
%\lhead{\textcolor{gray}{\it Christian Bird, Research Statement}}
%\rhead{\textcolor{gray}{\thepage/\totalpages{}}}
%\rhead{\thepage}
%\renewcommand{\headrulewidth}{0pt} 
%\renewcommand{\footrulewidth}{0pt} 
%\fancyfoot[C]{\footnotesize http://www.stanford.edu/$\sim$sundaes/application} 
%\ref{TotPages}

% This kind of makes 10pt to 9 pt.
%\begin{small}

%\vspace*{0.1cm}
\begin{center}
{\LARGE \bf Teaching Statement}\\
\vspace*{0.2cm}
{\large Christian Bird}\\
\vspace*{0.1cm}
\texttt{\normalsize (cabird@ucdavis.edu)}\\
\end{center}
%\vspace*{0.2cm}

%\begin{document}
%\centerline {\Large \bf Research Statement for Sundar Iyer}
%\vspace{0.5cm}

% Write about research interests...
%\footnotemark
%\footnotetext{Check This}

While I originally entered computer science because I found it interesting and
rewarding in its own right, I have had many opportunities for teaching along
the way and have found that I genuinely enjoy teaching things that I am
passionate about.  After having taken classes from many instructors, and
observing others, as well as attending talks, invited lectures, and keynotes, I
have learned that an audience is engaged and retains the most when the
person presenting material is enthusiastic and passionate about the subject.

My wife received an undergraduate degree in education and has been a teacher
for 6 years.  During my teaching responsibilities, I have had the benefit of
discussing strategies and methods with her.  For example, I've used in-class
discussions and exercises to address learning at the different level's of Bloom's
Taxonomy.  On one occasion, when a number of students were attending my office
hours due to confusion on the use of functional programming, I worked with one
student and then watched as she taught the concepts to other students who arrived
soon after.  Because she had to understand well enough to teach others,
\emph{and} because the other students asked insightful questions, all were able
to learn together in a memorable way.

I believe that understanding fundamentals is important, and
that an application of fundamentals aids students because it helps them
see how they are used in practice and most people learn better by doing than only
reading or listening.  I have experienced this in my own career.
As an example, I found DFAs and NFAs confusing when encountering them in my
freshman year of college.  However, after studying their uses and writing a
program to convert an NFA to a DFA and execute it on an input string, I
understood \emph{how} they worked, \emph{when} they could and should be used,
and \emph{why} one would want to convert an NFA to a DFA.  While not all
students learn in the same way, I plan to teach fundamentals to students and
follow up with practical exercises and discussion of current practice.

My research discipline is software engineering and I hope to teach in this
area.  As results from software engineering research transfer to practice in
industry, the set of skills that a software engineer needs is changing. Fresh
graduates entering the profession should be familiar not only with programming
and software development processes, but also empirical methods that can help
improve productivity and software quality.  Many recent developments are simple
to implement and can help in a software production environment such as:
identifying the most failure prone modules prior to deployment, determining
what communication channels should exist based team and software structure, and
identifying possible cross-cutting concerns early in the software development
life-cycle.  The Capability Maturity Model (CMM), a metric by which
software development organizations are measured, requires that processes and
outcomes be measured and analyzed.  These are skills that I acquired after
entering the workforce and beginning graduate school, but which \emph{should}
be taught at the undergraduate level.  Teaching these skills will help new
graduates differentiate themselves from other jb candidates and will raise
the quality of software engineering in general.

While I look forward to teaching classes on software engineering at the
graduate and undergraduate level, my interests and knowledge in computer
science is more broad.  My minor area in graduate school is in theory, and I
would enjoy and feel comfortable teaching classes on programming languages,
algorithms and data structures, and theory of computation as well as
introductory computer science courses.

I have been fortunate to work with an advisor who is not only a bright
researcher, but also a gifted mentor.  Prem has seen it as his responsibility
to help me become an independent and critical thinker.  While an advisor should
never need to micro-manage graduate students, the backgrounds of different
students is diverse and some will need more guidance up front than others.
Often, my advisor has acted as a wonderful motivator.  On multiple occasions
early in my career, I would arriave at a weekly meeting disappointed with
recent results of experiments or the feeling that we had been ``scooped''.  In
every case, I came away re-energized, seeing the situation in a new light, and
with new ideas to move forward with.  I now have a peer relationship with my
advisor and often debate the merits of different ideas, results, and approaches
as we discuss papers we have read or research plans for the future.  I plan to
use a similar approach as I advise graduate students.  Beyond assisting in
teaching university courses, I have had the opportunity to work with and mentor
both undergraduates and fellow graduate students who have worked with me on
research projects.  This process has been rewarding as I have been able to help
students as they tackle real world problems, face unforeseen challenges, and
finally write up their results for publication.  I hope to continue to guide
students in their own endeavors, whether they be careers in research or
practice.


\end{document}

